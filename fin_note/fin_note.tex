\documentclass[10pt, a4paper, landscape]{article}

%%%%%%%%%%%%%%%紙張大小設定%%%%%%%%%%%%%%%
% \paperwidth=65cm
% \paperheight=160cm

%%%%%%%%%%%%%%%引入Package%%%%%%%%%%%%%%%
\usepackage[margin=0.5cm]{geometry} % 上下左右距離邊緣2cm
\usepackage{mathtools,amsthm,amssymb} % 引入 AMS 數學環境
\usepackage{yhmath}      % math symbol
\usepackage{graphicx}    % 圖形插入用
\usepackage{fontspec}    % 加這個就可以設定字體
\usepackage{type1cm}	 % 設定fontsize用
\usepackage{titlesec}   % 設定section等的字體
\usepackage{titling}    % 加強 title 功能
\usepackage{fancyhdr}   % 頁首頁尾
\usepackage{tabu}   % 加強版 table
\usepackage[square, comma, numbers, super, sort&compress]{natbib}
% cite加強版
\usepackage[unicode=true, pdfborder={0 0 0}, bookmarksdepth=-1]{hyperref}
% ref加強版
\usepackage[usenames, dvipsnames]{color}  % 可以使用顏色
\usepackage[shortlabels]{enumitem}  % 加強版enumerate
\usepackage{xpatch}

\usepackage{multirow}   % multirow
% \usepackage{soul}       % highlight
% \usepackage{ulem}       % 字加裝飾
% \usepackage{wrapfig}     % 文繞圖
% \usepackage{floatflt}    % 浮動 figure
% \usepackage{float}       % 浮動環境
% \usepackage{caption}    % caption 增強
% \usepackage{subcaption}    % subfigures
% \usepackage{setspace}    % 控制空行
% \usepackage{mdframed}   % 可以加文字方框
% \usepackage{multicol}   % 多欄
% \usepackage{siunitx}      % SI unit

%%%%%%%%%%%%%%%%%%%TikZ%%%%%%%%%%%%%%%%%%%%%%
% \usepackage{tikz}

%%%%%%%%%%%%%%中文 Environment%%%%%%%%%%%%%%%
\usepackage[CheckSingle, CJKmath]{xeCJK}  % xelatex 中文
\usepackage{CJKulem}	% 中文字裝飾
\setCJKmainfont[BoldFont=cwTeX Q Hei]{cwTeX Q Ming}
% 設定中文為系統上的字型,而英文不去更動,使用原TeX字型

% \XeTeXlinebreaklocale "zh"             %這兩行一定要加,中文才能自動換行
% \XeTeXlinebreakskip = 0pt plus 1pt     %這兩行一定要加,中文才能自動換行

%%%%%%%%%%%%%%%字體大小設定%%%%%%%%%%%%%%%
% \def\normalsize{\fontsize{10}{15}\selectfont}
% \def\large{\fontsize{40}{60}\selectfont}
% \def\Large{\fontsize{50}{75}\selectfont}
% \def\LARGE{\fontsize{90}{20}\selectfont}
% \def\huge{\fontsize{34}{51}\selectfont}
% \def\Huge{\fontsize{38}{57}\selectfont}

%%%%%%%%%%%%%%%Theme Input%%%%%%%%%%%%%%%%
% \input{themes/chapter/neat}
% \input{themes/env/problist}

%%%%%%%%%%%titlesec settings%%%%%%%%%%%%%%
% \titleformat{\chapter}{\bf\Huge}
            % {\arabic{section}}{0em}{}
% \titleformat{\section}{\centering\Large}
            % {\arabic{section}}{0em}{}
% \titleformat{\subsection}{\large}
            % {\arabic{subsection}}{0em}{}
% \titleformat{\subsubsection}{\bf\normalsize}
            % {\arabic{subsubsection}}{0em}{}
% \titleformat{command}[shape]{format}{label}
            % {編號與標題距離}{before}[after]

%%%%%%%%%%%%variable settings%%%%%%%%%%%%%%
% \numberwithin{equation}{section}
% \setcounter{secnumdepth}{4}  %章節標號深度
% \setcounter{tocdepth}{1}  %目錄深度
% \graphicspath{{images/}}  % 搜尋圖片目錄

%%%%%%%%%%%%%table settings%%%%%%%%%%%%%%%%
\newcolumntype{C}[1]{>{\centering\arraybackslash}p{#1}}
\newcolumntype{'}{!{\vrule width 1pt}}
\newcommand{\tstack}[2][c]{\begin{tabular}{@{}c@{}}#2\end{tabular}}
\def\arraystretch{1.3}

%%%%%%%%%%%%%%%頁面設定%%%%%%%%%%%%%%%
\setlength{\headheight}{15pt}  %with titling
\setlength{\droptitle}{-1.5cm} %title 與上緣的間距
\parindent=0pt %設定縮排的距離
% \parskip=1ex  %設定行距
% \pagestyle{empty}  % empty: 無頁碼
% \pagestyle{fancy}  % fancy: fancyhdr

% use with fancygdr
% \lhead{\leftmark}
% \chead{}
% \rhead{}
% \lfoot{}
% \cfoot{}
% \rfoot{\thepage}
% \renewcommand{\headrulewidth}{0.4pt}
% \renewcommand{\footrulewidth}{0.4pt}

% \fancypagestyle{firststyle}
% {
  % \fancyhf{}
  % \fancyfoot[C]{\footnotesize Page \thepage\ of \pageref{LastPage}}
  % \renewcommand{\headrule}{\rule{\textwidth}{\headrulewidth}}
% }

%%%%%%%%%%%%%%%重定義一些command%%%%%%%%%%%%%%%
\renewcommand{\contentsname}{目錄}  %設定目錄的標題名稱
\renewcommand{\refname}{參考資料}  %設定參考資料的標題名稱
\renewcommand{\abstractname}{\LARGE Abstract} %設定摘要的標題名稱

%%%%%%%%%%%%%%%特殊功能函數符號設定%%%%%%%%%%%%%%%
% \newcommand{\citet}[1]{\textsuperscript{\cite{#1}}}
\DeclarePairedDelimiter{\abs}{\lvert}{\rvert}
\DeclarePairedDelimiter{\norm}{\lVert}{\rVert}
\DeclarePairedDelimiter{\inpd}{\langle}{\rangle} % inner product
\DeclarePairedDelimiter{\ceil}{\lceil}{\rceil}
\DeclarePairedDelimiter{\floor}{\lfloor}{\rfloor}
\DeclareMathOperator{\adj}{adj}
\DeclareMathOperator{\sech}{sech}
\DeclareMathOperator{\csch}{csch}
\DeclareMathOperator{\arcsec}{arcsec}
\DeclareMathOperator{\arccot}{arccot}
\DeclareMathOperator{\arccsc}{arccsc}
\DeclareMathOperator{\arccosh}{arccosh}
\DeclareMathOperator{\arcsinh}{arcsinh}
\DeclareMathOperator{\arctanh}{arctanh}
\DeclareMathOperator{\arcsech}{arcsech}
\DeclareMathOperator{\arccsch}{arccsch}
\DeclareMathOperator{\arccoth}{arccoth}
\newcommand{\np}[1]{\\[{#1}] \indent}
\newcommand{\transpose}[1]{{#1}^\mathrm{T}}
%%%% Geometry Symbol %%%%
\newcommand{\degree}{^\circ}
\newcommand{\Arc}[1]{\wideparen{{#1}}}
\newcommand{\Line}[1]{\overleftrightarrow{{#1}}}
\newcommand{\Ray}[1]{\overrightarrow{{#1}}}
\newcommand{\Segment}[1]{\overline{{#1}}}

%%%%%%%%%%%%%%%證明、結論、定義等等的環境%%%%%%%%%%%%%%%
\renewcommand{\proofname}{\bf 證明:} %修改Proof 標頭
\newtheoremstyle{mystyle}% 自定義Style
  {6pt}{15pt}%       上下間距
  {}%               內文字體
  {}%               縮排
  {\bf}%            標頭字體
  {.}%              標頭後標點
  {1em}%            內文與標頭距離
  {}%               Theorem head spec (can be left empty, meaning 'normal')

% 改用粗體,預設 remark style 是斜體
\theoremstyle{mystyle}	% 定理環境Style
\newtheorem{theorem}{定理}
\newtheorem{definition}{定義}
\newtheorem{formula}{公式}
\newtheorem{condition}{條件}
\newtheorem{supposition}{假設}
\newtheorem{conclusion}{結論}
\newtheorem{lemma}{引理}
\newtheorem{property}{性質}

%%%%%%%%%%%%%%%Title的資訊%%%%%%%%%%%%%%%
\title{Signals and Systems Final Notes} %標題
\author{LeoMao} %作者
\date{\today} %日期

\begin{document}
% \maketitle %製作tilte page
% \thispagestyle{empty}  %去除頁碼
% \thispagestyle{fancy}  %使用fancyhdr
% \tableofcontents %目錄
%%%%%%%%%%%%%%%%%%%include file here%%%%%%%%%%%%%%%%%%%%%%%%%
\begin{minipage}{12cm}
\begin{tabu}{C{2.5cm}'C{2.5cm}C{2.5cm}C{2.5cm}}
  property & $x(t)$ & $X(s)$ & ROC \\
  \hline
  definition &
  \multicolumn{2}{c}{\tstack{
  $\displaystyle X(s) = \int_{-\infty}^{\infty} x(t)e^{-st} dt$ \\
  $\displaystyle x(t) = \frac{1}{2\pi j}\int_{\sigma-j\infty}^{\sigma+j\infty}
  X(s)e^{st}ds$
  }} & $R$
  \\
  time shift & $x(t-t_0)$ & $e^{-st_0}X(s)$ & $R$
  \\
  s-Domain shift & $e^{s_0t}x(t)$ & $X(s-s_0)$ & $R - s_0$
  \\
  scaling & $x(at)$ & $\frac{1}{\abs{a}}X\left(\frac{s}{a}\right)$ &
  $aR$
  \\
  conjugation & $x^*(t)$ & $X^*(s^*)$ & $R$
  \\
  convolution & $x_1(t)\ast x_2(t)$ & $X_1(s)X_2(s)$ &
  At least $R_1 \cap R_2$
  \\
  time 微分 & $\frac{d}{dt}x(t)$ & $s X(s)$ & At least $R$
  \\
  s 微分 & $-tx(t)$ & $\frac{d}{ds}X(s)$ & $R$
  \\
  time 積分 & $\displaystyle \int_{-\infty}^{t}x(\tau)d\tau$ &
  $\frac{1}{s}X(s)$ & At least $R \cap \mathfrak{Re}\{s\} > 0$
  \\
  \hline
\end{tabu}
\begin{tabu}{C{4cm}C{3cm}C{3cm}}
  Signal & $\mathcal{L}$ & ROC \\ \hline
  $\delta(t-T)$ & $e^{-sT}$ & All $s$ \\
  $\displaystyle \frac{t^{n-1}}{(n-1)!}e^{-\alpha t}u(t)$ &
  $\displaystyle \frac{1}{(s+\alpha)^n}$ &
  $\mathfrak{Re}\{s\} > -\alpha$ \\
  $\displaystyle -\frac{t^{n-1}}{(n-1)!}e^{-\alpha t}u(-t)$ &
  $\displaystyle \frac{1}{(s+\alpha)^n}$ &
  $\mathfrak{Re}\{s\} < -\alpha$ \\
  $e^{-\alpha t} \cos(\omega_0 t)u(t)$ &
  $\displaystyle \frac{s+\alpha}{(s+\alpha)^2 + \omega_0^2}$ &
  $\mathfrak{Re}\{s\} > -\alpha$ \\
  $e^{-\alpha t} \sin(\omega_0 t)u(t)$ &
  $\displaystyle \frac{\omega_0}{(s+\alpha)^2 + \omega_0^2}$ &
  $\mathfrak{Re}\{s\} > -\alpha$ \\
  $\displaystyle u_n(t) = \frac{d^n\delta(t)}{dt^n}$ &
  $s^n$ & All $s$ \\
  $u_{-n}(t) = \underbrace{u(t) \ast \dots \ast u(t)}_\text{$n$ times}$ &
  $\displaystyle \frac{1}{s^n}$ & All $s$ \\
  \hline
\end{tabu}
Linearity: ROC 至少包含全部 ROC 的交集。\\
Causality: ROC 包含右半 s-plane,
Stablity: ROC 包含 $\sigma=0$ 那條 \\
Unilateral Laplace Transform special:
$\frac{d}{dt}x(t) \Rightarrow sX(s) - x(0^-)$ \\
Initial Value Theorem:\\
if $x(t) = 0$ for $t<0$ and no impulses or higher-order singularities
at $t=0$, \\
$x(0^+) = \lim_{s\to\infty} sX(s)$ \\ 
Final Value Theorem:\\
if $x(t) = 0$ for $t<0$ and has a finite limit
as $t \to \infty$, \\
$\lim_{t\to\infty}x(t) = \lim\limits_{s\to 0} sX(s)$
\end{minipage}
\begin{minipage}{13cm}
\begin{tabu}{C{3cm}'C{2.5cm}C{2.5cm}C{4cm}}
  property & $x[n]$ & $X(z)$ & ROC \\
  \hline
  definition &
  \multicolumn{2}{c}{\tstack{
  $\displaystyle X(z) = \sum_{n=-\infty}^{\infty} x[n]z^{-n}$ \\
  $\displaystyle x[n] = \frac{1}{2\pi j}\oint_r X(z)z^{n-1}dz$
  }} & $R$
  \\
  time shift & $x[n-n_0]$ & $z^{-n_0}X(z)$ & $R$, except for the
  possible addition or deletion of the origin
  \\
  z-Domain scaling & $z_0^nx[n]$ & $X\left(\frac{z}{z_0}\right)$ &
  $z_0R$
  \\
  time reversal & $x[-n]$ & $X(z^{-1})$ & $R^{-1}$
  \\
  time expansion & $x_{(k)}[n]$ & $X(z^k)$ & $R^{1/k}$
  \\
  conjugation & $x^*[n]$ & $X^*(z^*)$ & $R$ 
  \\
  convolution & $x_1[n]\ast x_2[n]$ & $X_1(z)X_2(z)$ &
  At least $R_1 \cap R_2$
  \\
  z 微分 & $nx[n]$ & $-z\frac{d}{dz}X(z)$ & $R$
  \\
  time 積分 & $\displaystyle \sum_{k=-\infty}^{n}x[k]$ &
  $\frac{1}{1 - z^{-1}}X(s)$ & At least $R \cap \{\abs{z} > 1\}$
  \\
  \hline
\end{tabu}
\begin{tabu}{C{4cm}C{4cm}C{4cm}}
  Signal & $\mathcal{Z}$ & ROC \\ \hline
  $\delta[n-m]$ & $z^{-m}$ & All $z$, except $0$ (if $m>0$) or
  $\infty$ (if $m<0$) \\
  $\alpha^nu[n]$ & $\displaystyle \frac{1}{1-\alpha z^{-1}}$ &
  $\abs{z} > \abs{\alpha}$ \\
  $-\alpha^nu[-n-1]$ & $\displaystyle \frac{1}{1-\alpha z^{-1}}$ &
  $\abs{z} < \abs{\alpha}$ \\
  $n\alpha^nu[n]$ &
  $\displaystyle \frac{\alpha z^{-1}}{(1-\alpha z^{-1})^2}$ &
  $\abs{z} > \abs{\alpha}$ \\
  $-n\alpha^nu[-n-1]$ &
  $\displaystyle \frac{\alpha z^{-1}}{(1-\alpha z^{-1})^2}$ &
  $\abs{z} < \abs{\alpha}$ \\
  $r^n \cos(\omega_0 n)u[n]$ &
  $\displaystyle \frac{1-r\cos(\omega_0)z^{-1}}
  {1-2\cos(\omega_0)z^{-1} + r^2 z^{-2}}$ &
  $\abs{z} > r$ \\
  $r^n \sin(\omega_0 n)u[n]$ &
  $\displaystyle \frac{1-r\sin(\omega_0)z^{-1}}
  {1-2\cos(\omega_0)z^{-1} + r^2 z^{-2}}$ &
  $\abs{z} > r$ \\
  \hline
\end{tabu}
Linearity: ROC 至少包含全部 ROC 的交集。\\
Causality: ROC 包含外圍至 $\infty$ ,
Stablity: ROC 包含 $r = 1$ 那圈 \\
Unilateral Z-Transform special:
$x[n-1] \Rightarrow z^{-1}X(z) + x[-1],\, x[n+1] \Rightarrow zX(z) - zx[0]$
\\
Initial Value Theorem:\\
if $x[n] = 0$ for $n<0$, $x[n] = \lim_{z\to\infty} X(z)$ \\ 
\end{minipage}
\newpage
\begin{center}
\begin{tabu}{C{2cm}'C{2.8cm}C{2.8cm}'C{2.8cm}C{2.8cm}'C{2.8cm}C{2.8cm}'C{2.8cm}C{2.8cm}'}
  property & \tstack{無週期\\$x(t)$} & $X(j\omega)$ &
  \tstack{無週期\\$x[n]$} & \tstack{$X(e^{j\omega})$\\$T=2\pi$} &
  \tstack{$x(t)$週期$T$\\$\omega_0=\frac{2\pi}{T}$} & $a_k$ &
  \tstack{$x[n]$週期$N$\\$\omega_0=\frac{2\pi}{N}$} &
  \tstack{$a_k$\\$T = N$} \\
  \hline
  Def. &
  \multicolumn{2}{c'}{\tstack{
  $\displaystyle X(j\omega) = \int_{-\infty}^{\infty} x(t)e^{-j\omega t} dt$ \\
  $\displaystyle x(t) = \frac{1}{2\pi}\int_{-\infty}^{\infty} X(j\omega)e^{j\omega t}d\omega$
  }} &
  \multicolumn{2}{c'}{\tstack{
  $\displaystyle X(e^{j\omega}) = \sum_{n=-\infty}^{\infty} x[n]e^{-j\omega n}$ \\
  $\displaystyle x[n] = \frac{1}{2\pi}\int_{2\pi} X\left(e^{j\omega}\right)e^{j\omega n} d\omega$
  }} &
  \multicolumn{2}{c'}{\tstack{
  $\displaystyle a_k = \frac{1}{T}\int_{T} x(t)e^{-jk\omega_0 t} dt$ \\
  $\displaystyle x(t) = \sum_{k=-\infty}^{\infty} a_k e^{jk\omega_0 t}$
  }} &
  \multicolumn{2}{c'}{\tstack{
  $\displaystyle a_k = \frac{1}{N}\sum_{n=\inpd{N}} x[n]e^{-jk\omega_0 n}$ \\
  $\displaystyle x[n] = \sum_{k=\inpd{N}} a_k e^{jk\omega_0 n}$
  }}
  \\
  \hline
  time 微分 & $x'(t)$ & $j\omega X(j\omega)$ &
  $x[n] - x[n-1]$ & $\left(1-e^{-jw}\right) X\left(e^{j\omega}\right)$ &
  $x'(t)$ & $jk\omega_0 a_k$ &
  $x[n] - x[n-1]$ & $\left(1-e^{-jw}\right) a_k$ \\
  \cline{2-9}
  freq. 微分 & $tx(t)$ & $j \frac{dX(j\omega)}{d\omega}$ &
  $nx[n]$ & $j \frac{dX\left(e^{j\omega}\right)}{d\omega}$ &
  \multirow{3}{*}{\tstack{
  $\displaystyle \int_{-\infty}^t x(\tau)d\tau$\\$a_0$ must be $0$
  }} &
  \multirow{3}{*}{$\displaystyle \frac{1}{j\omega_0 k}a_k$} &
  \multirow{3}{*}{\tstack{
  $\displaystyle \sum_{k=-\infty}^n x(k)$\\$a_0$ must be $0$
  }} &
  \multirow{3}{*}{$\displaystyle \frac{1}{1 - e^{-j\omega_0 k}}a_k$} \\
  \cline{2-5}
  time 積分 & $\displaystyle \int_{-\infty}^{t}x(\tau)d\tau$ &
  $\frac{1}{j\omega}X(j\omega)+\pi X(0)\delta(\omega)$ &
  $\displaystyle \sum_{k=-\infty}^{n}x[k]$ &
  $\frac{1}{1-e^{-j\omega}}X\left(e^{j\omega}\right) +
   \pi X\left(e^{j0}\right) \delta_{2\pi}(\omega)$ &
   & &
   &  \\
  \cline{2-5}
  freq. 積分 &
  $\frac{-1}{jt}x(t)+\pi x(0)\delta(t)$ & $\int_{-\infty}^{\omega}X(j\eta)d\eta$ &
  $\frac{j}{n}x[n]$ & $j \int_{-\pi}^{\omega}X\left(e^{j\eta}\right)d\eta$ &
   & &
   &  \\
  \hline
  Duality &
  \multicolumn{2}{c'}{
  $\displaystyle y(t) \xleftrightarrow{\mathcal{F}} z(\omega)
  \Leftrightarrow z(t) \xleftrightarrow{\mathcal{F}} 2\pi y(-\omega)$
  } &
  \multicolumn{4}{c'}{
  $\displaystyle y[n] \xleftrightarrow{\mathcal{F}} z(\omega)
  \Leftrightarrow z(t) \xleftrightarrow{\mathcal{FS}} y[-k]$
  } &
  \multicolumn{2}{c'}{
  $\displaystyle y[n] \xleftrightarrow{\mathcal{FS}} z[k]
  \Leftrightarrow z[n] \xleftrightarrow{\mathcal{FS}} \frac{1}{N} y[-k]$
  } \\
  \hline
  Parseval's Theorem &
  \multicolumn{2}{c'}{
  $\displaystyle \int_{-\infty}^{\infty}\abs*{x(t)}^2 dt =
  \frac{1}{2\pi}\int_{-\infty}^{\infty} \abs*{X(j\omega)}^2 d\omega$
  } &
  \multicolumn{2}{c'}{
  $\displaystyle \sum_{n=-\infty}^{\infty}\abs*{x[n]}^2 =
  \frac{1}{2\pi}\int_{2\pi} \abs*{X\left(e^{j\omega}\right)}^2 d\omega$
  } &
  \multicolumn{2}{c'}{
  $\displaystyle \frac{1}{T}\int_{T} \abs*{x(t)}^2 dt =
   \sum_{k=-\infty}^{\infty}\abs*{a_k}^2 $
  } &
  \multicolumn{2}{c'}{
  $\displaystyle \frac{1}{N}\sum_{n=\inpd{N}}\abs*{x[n]}^2 =
  \sum_{k=\inpd{N}}\abs*{a_k}^2 $
  }
\end{tabu}
\end{center}
\begin{tabu}{C{4cm}C{4.5cm}}
  signal & $\mathcal{F}$ \\ \hline
  $e^{j\omega_0t}$ & $2\pi\delta(\omega-\omega_0)$ \\ \hline
  \tstack{$x(t) = \begin{cases} 1, & \abs{t} < T_1 \\
  0, & T_1 < \abs{t} \le \frac{T}{2} \end{cases} $ \\
  $x(t + T) = x(t)$
  } &
  $\sum \limits_{k=-\infty}^{\infty}\frac{2\sin(k\omega_0 T_1)}{k}
  \delta(\omega-k\omega_0)$ \\ \hline
  $\delta_T(t)$ &
  $\omega_0 \delta_{\omega_0}(\omega)$ \\ \hline
  $x(t) = \begin{cases} 1, & \abs{t} \le T_1 \\
  0, & \abs{t} > T_1 \end{cases} $ &
  $\displaystyle \frac{2\sin(\omega T_1)}{\omega}$ \\ \hline
  $\displaystyle \frac{\sin(\omega_0 t)}{\pi t}$ &
  $X(j\omega) = \begin{cases} 1, & \abs{\omega} < \omega_0 \\
  0, & \abs{\omega} > \omega_0 \end{cases} $ \\ \hline
  $\displaystyle \frac{t^{n-1}}{(n-1)!}e^{-at}u(t)$ &
  $\displaystyle \frac{1}{(a+j\omega)^n}$ \\ \hline
  $\displaystyle e^{-\abs{at}}$ &
  $\displaystyle \frac{2a}{a^2+\omega^2}$ 
\end{tabu}
\begin{tabu}{C{4cm}C{6cm}C{6.5cm}}
  signal & $\mathcal{F}$ & $\mathcal{FS}$ \\ \hline
  $e^{j\omega_0n}$ & $2\pi\delta_{2\pi}(\omega-\omega_0)$ &
  \tstack{
  $\omega_0 = \frac{2\pi m}{N} \Rightarrow a_k = \begin{cases}
    1, & k \equiv m \mod N \\ 0, & \text{otherwise}
  \end{cases}$ \\
  $\frac{\omega_0}{2\pi}$ 無理 $\Rightarrow$ aperiodic
  } \\ \hline
  \tstack{$ x[n] = \begin{cases} 1, & \abs{n} < N_1 \\
  0, & N_1 < \abs{n} \le \floor*{\frac{N}{2}} \end{cases} $ \\
  $x[n + N] = x[n]$ ($N$ is odd)
  } &
  $2\pi\sum \limits_{k=-\infty}^{\infty}a_k\delta(\omega-k\omega_0)$ & 
  $a_k = \begin{cases}
    \frac{\sin\left(k\omega_0\left(N_1+\frac{1}{2}\right)\right)}{N\sin(k\omega_0)},
    & k \not\equiv 0 \mod N \\
    \frac{2N_1+1}{N}, & k \equiv 0 \mod N
  \end{cases}$
  \\ \hline
  $\delta_N[n]$ &
  $\omega_0 \delta_{\omega_0}(\omega)$ & $a_k = \frac{1}{N}$ \\ \hline
  $x[n] = \begin{cases} 1, & \abs{n} \le N_1 \\
  0, & \abs{n} > N_1 \end{cases} $ &
  $\displaystyle
  \frac{\sin\left(\omega\left(N_1+\frac{1}{2}\right)\right)}{\sin\left(\frac{\omega}{2}\right)}$ & \\ \hline
  $\frac{\sin(\omega_1n)}{\pi n}, 0 < \omega_1 < \pi$ &
  $X\left(e^{j\omega}\right) = \begin{cases}
    1, & \abs{\omega}  \le \omega_1 \\
    0, & \omega_1 < \abs{\omega} \le \pi
  \end{cases}$ & \\ \hline
  $\frac{(n+r-1)!}{n!(r-1)!}a^nu[n]$ &
  $\frac{1}{\left(1 - ae^{-j\omega}\right)^r}$
\end{tabu}
\\
{\bf 重要}:本表為省空間,令
$\delta_\alpha(t) = \sum\limits_{k=-\infty}^{\infty} \delta(t-k\alpha)
\quad \delta_m[n] = \sum\limits_{k=-\infty}^{\infty} \delta[n-km]$ \\
%%%%%%%%%%%%%%%%%%%%%%%%%%%%%%%%%%%%%%%%%%%%%%%%%%%%%%%%%%%%%
% \bibliographystyle{plain}
% \begin{thebibliography}{99}
% \bibitem[1]{ex}\verb|http://www.example.com/|
% \end{thebibliography}
\end{document}
